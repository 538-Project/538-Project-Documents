%
% File naacl2019.tex
%
%% Based on the style files for ACL 2018 and NAACL 2018, which were
%% Based on the style files for ACL-2015, with some improvements
%%  taken from the NAACL-2016 style
%% Based on the style files for ACL-2014, which were, in turn,
%% based on ACL-2013, ACL-2012, ACL-2011, ACL-2010, ACL-IJCNLP-2009,
%% EACL-2009, IJCNLP-2008...
%% Based on the style files for EACL 2006 by 
%%e.agirre@ehu.es or Sergi.Balari@uab.es
%% and that of ACL 08 by Joakim Nivre and Noah Smith

\documentclass[11pt,a4paper]{article}
\usepackage[hyperref]{naaclhlt2019}
\usepackage{times}
\usepackage{latexsym}

\usepackage{url}

%\aclfinalcopy % Uncomment this line for the final submission
%\def\aclpaperid{***} %  Enter the acl Paper ID here

%\setlength\titlebox{5cm}
% You can expand the titlebox if you need extra space
% to show all the authors. Please do not make the titlebox
% smaller than 5cm (the original size); we will check this
% in the camera-ready version and ask you to change it back.

\newcommand\BibTeX{B{\sc ib}\TeX}

\title{Proposal: Your Project Title}

\author{First Author \\
  Affiliation / Address line 1 \\
  Affiliation / Address line 2 \\
  Affiliation / Address line 3 \\
  {\tt email@domain} \\\And
  Second Author \\
  Affiliation / Address line 1 \\
  Affiliation / Address line 2 \\
  Affiliation / Address line 3 \\
  {\tt email@domain} \\}

\date{}

\begin{document}
\maketitle
%\begin{abstract}

%A one paragraph description of your project including the key outcomes.

%\end{abstract}


\section{Proposal Description}

Paragraph 1: What is the overall objective of your project? That is what problem are you solving. Why should anyone care about this problem?
Why is this problem difficult to solve?\\

\textbf{NEEDED FOR PROPOSAL}

Paragraph 2: What are the broad approaches that are used to address this problem? Please find three papers that are relevant for this problem and summarize the main idea behind each paper in one line.\\

\textbf{OPTIONAL FOR PROPOSAL}
%Paragraph 3: What is a specific issue or gap that exists in these above approaches? Give an example to illustrate this gap.\\

Paragraph 3: What are the ideas/methods you are trying to address this problem? How do these relate to the ideas discussed in the paper.  \\

\textbf{NEEDED FOR PROPOSAL}

Paragraph 5: Briefly mention how you will test these ideas. Mention the system that you are starting as your baseline system. 
Then say what other systems or variations of the system you are trying out.

\textbf{NEEDED FOR PROPOSAL}

Paragraph 5: How will you evaluate your ideas? State how you will do this. Below is a template but you dont have to follow it. The idea however is for you think about your evaluation in this type of a structured format. You will have to think carefully about what you will evaluate and how.

\textbf{NEEDED FOR PROPOSAL}

\begin{itemize}
\item Research question 1: Can doing [insert your idea 1 here] help improve performance on the task? 
Show an empty table that will present the main results you aim to obtain. The table will typically have rows for the baselines, your models/variations, and columns that present evaluation metrics (e.g. classification accuracy, precision, recall, F1).
The numbers of course will be empty at this stage.

\item Research question 2: When does [insert your idea 1 here] help improve performance on the task? 
Describe what kind of analysis you will do to figure out in what situations your idea is helping. This will usually amount to looking at the cases where your model worked and coming up with hypotheses for why. Sometimes you might be able to categorize the instances based on some aspect that is meaningful for your problem (e.g. the length of the text, the presence of certain types of entities etc).

\item Research question 3: When does [insert your idea 1 here] hurt model performance?
Describe what kind of analysis you will do to figure out in what situations your idea is hurting. This will usually amount to looking at the cases where your model failed and coming up with hypotheses for why. Sometimes you might be able to categorize the instances based on some aspect that is meaningful for your problem (e.g. the length of the text, the presence of certain types of entities etc).


\end{itemize}


Paragraph 6: What are the main outcomes you hope of this project? Give a short list of what your observations were based on the evaluation.
\textbf{OPTIONAL FOR PROPOSAL}
And list the what you learned by doing this project. For example:
\begin{enumerate}
  
\item We implemented X for the task of ...
\item We developed a new addition Y to X
\item Our evaluation shows Y performs better by Z F1 points.
\item Analysis shows Y works well when ... and fails when ...
\item Based on our work we conclude that investigating P, Q, and R might be promising avenues for future work. 
\end{enumerate}

\newpage
\end{document}